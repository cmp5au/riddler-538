% !TEX TS-program = pdflatex
% !TEX encoding = UTF-8 Unicode

% This is a simple template for a LaTeX document using the "article" class.
% See "book", "report", "letter" for other types of document.

\documentclass[11pt]{article} % use larger type; default would be 10pt

\usepackage[utf8]{inputenc} % set input encoding (not needed with XeLaTeX)

%%% Examples of Article customizations
% These packages are optional, depending whether you want the features they provide.
% See the LaTeX Companion or other references for full information.

%%% PAGE DIMENSIONS
\usepackage{geometry} % to change the page dimensions
\geometry{a4paper} % or letterpaper (US) or a5paper or....
% \geometry{margin=2in} % for example, change the margins to 2 inches all round
% \geometry{landscape} % set up the page for landscape
%   read geometry.pdf for detailed page layout information

\usepackage{graphicx} % support the \includegraphics command and options

% \usepackage[parfill]{parskip} % Activate to begin paragraphs with an empty line rather than an indent

%%% PACKAGES
\usepackage{amsfonts}
\usepackage{amsmath}
\usepackage[backref,pdfpagemode=FullScreen,colorlinks=true]{hyperref}
\usepackage{booktabs} % for much better looking tables
\usepackage{array} % for better arrays (eg matrices) in maths
\usepackage{paralist} % very flexible & customisable lists (eg. enumerate/itemize, etc.)
\usepackage{verbatim} % adds environment for commenting out blocks of text & for better verbatim
\usepackage{subfig} % make it possible to include more than one captioned figure/table in a single float
% These packages are all incorporated in the memoir class to one degree or another...

%%% HEADERS & FOOTERS
\usepackage{fancyhdr} % This should be set AFTER setting up the page geometry
\pagestyle{plain} % options: empty , plain , fancy
\renewcommand{\headrulewidth}{0pt} % customise the layout...
\lhead{}\chead{}\rhead{}
\lfoot{}\cfoot{\thepage}\rfoot{}

%%% SECTION TITLE APPEARANCE
\usepackage{sectsty}
\allsectionsfont{\sffamily\mdseries\upshape} % (See the fntguide.pdf for font help)
% (This matches ConTeXt defaults)

%%% ToC (table of contents) APPEARANCE
\usepackage[nottoc,notlof,notlot]{tocbibind} % Put the bibliography in the ToC
\usepackage[titles,subfigure]{tocloft} % Alter the style of the Table of Contents
\renewcommand{\cftsecfont}{\rmfamily\mdseries\upshape}
\renewcommand{\cftsecpagefont}{\rmfamily\mdseries\upshape} % No bold!

%%%Optionally increase matrix line spacing
\makeatletter
\renewcommand*\env@matrix[1][\arraystretch]{%
  \edef\arraystretch{#1}%
  \hskip -\arraycolsep
  \let\@ifnextchar\new@ifnextchar
  \array{*\c@MaxMatrixCols c}}
\makeatother

%%% END Article customizations

%%% The "real" document content comes below...

\title{538 Riddler: 1 July 2022}
\author{Colin Parker}

\begin{document}
\maketitle

\section{Express}

If $r_n$ is the amount you have remaining to complete with $n$ days remaining, then the amount you complete on this day
is given by $\frac{2}{n} \cdot r_n$. You complete the task if and only if the amount you complete is at least
the amount remaining:

$$\frac{2}{n} \cdot r_n \geq r_n \iff n \leq 2$$

The task would be completed on the second-to-last day of the year, December 30.

\section{Classic}

This riddle doesn't have a definitive answer, so we will examine the general behavior and approximate the
correct answer in the limit of a very large planet.

To see that there is no one answer, imagine that one of the cities
is polar and the other is equatorial, meaning that the radius of the planet is $R = 100(1 + \sqrt2)$ meters (small planet!).
In this case, you
can never climb high enough to see your friend on the ground floor. In the other limit, the planet is nearly locally flat,
so that it is not a problem to climb high enough to see your friend (assuming the tower is tall enough).

If we consider the great circle that contains both of these cities, call the minor arc angle of this great circle $\theta$,
and call the radius of the planet $R$ (measured in meters), then we can just barely see our friend if the line joining our
two points is tangent to the sphere of the planet at its midpoint, giving:

\begin{align*}
(R + 100)\cos \left(\frac{\theta}{2}\right) = R \\
R = \frac{100\cos \left(\frac{\theta}{2}\right)}{1 - \cos \left(\frac{\theta}{2}\right)}
\end{align*}

Once our friend climbs down to the bottom floor of his tower, let $h$ be the required height in meters in the tower for us to
just barely make out our friend again. Then the line joining us will be tangent to the sphere of the planet at our
friend's location, giving:

\begin{align*}
(R + h)\cos\theta & = R \\
h & = R\left(\frac{1 - \cos\theta}{\cos\theta}\right) \\
& = \frac{100 \cos \left(\frac{\theta}{2}\right)(1 - \cos\theta)}{\cos\theta (1 - \cos \left(\frac{\theta}{2}\right))} \\
& = \frac{400R^2 + 20000R}{R^2 - 200R - 10000} \text{, using the relation between } R \text{ and } \theta \text{ above.}
\end{align*}

As expected, this last expression has a vertical asymptote at $R = 100(1 + \sqrt2)$, and this expression is able to
easily give limits:

$$\lim_{\theta \to 0} h = \lim_{R \to \infty} h = \lim_{R \to \infty} \frac{400R^2 + 20000R}{R^2 - 200R - 10000} = 400$$

This is a plausible answer, because we are told the cities are neighboring, meaning that $\theta$ should be small.
However, we do have a precise possibility for the radius of the planet -- what if it wasn't a dream, and the radius
of the planet is exactly \href{https://fivethirtyeight.com/features/can-you-reach-the-target-number/}{8 megameters}?

Substituting $R = 8000000$ gives $\theta \approx 0.01$ radians and $h \approx 400.0125$ meters, showing that
400 meters is indeed a very reasonable estimation.

\end{document}
