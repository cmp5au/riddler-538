% !TEX TS-program = pdflatex
% !TEX encoding = UTF-8 Unicode

% This is a simple template for a LaTeX document using the "article" class.
% See "book", "report", "letter" for other types of document.

\documentclass[11pt]{article} % use larger type; default would be 10pt

\usepackage[utf8]{inputenc} % set input encoding (not needed with XeLaTeX)

%%% Examples of Article customizations
% These packages are optional, depending whether you want the features they provide.
% See the LaTeX Companion or other references for full information.

%%% PAGE DIMENSIONS
\usepackage{geometry} % to change the page dimensions
\geometry{a4paper} % or letterpaper (US) or a5paper or....
% \geometry{margin=2in} % for example, change the margins to 2 inches all round
% \geometry{landscape} % set up the page for landscape
%   read geometry.pdf for detailed page layout information

\usepackage{graphicx} % support the \includegraphics command and options

% \usepackage[parfill]{parskip} % Activate to begin paragraphs with an empty line rather than an indent

%%% PACKAGES
\usepackage{amsfonts}
\usepackage{amsmath}
\usepackage{booktabs} % for much better looking tables
\usepackage{array} % for better arrays (eg matrices) in maths
\usepackage{paralist} % very flexible & customisable lists (eg. enumerate/itemize, etc.)
\usepackage{verbatim} % adds environment for commenting out blocks of text & for better verbatim
\usepackage{subfig} % make it possible to include more than one captioned figure/table in a single float
% These packages are all incorporated in the memoir class to one degree or another...

%%% HEADERS & FOOTERS
\usepackage{fancyhdr} % This should be set AFTER setting up the page geometry
\pagestyle{plain} % options: empty , plain , fancy
\renewcommand{\headrulewidth}{0pt} % customise the layout...
\lhead{}\chead{}\rhead{}
\lfoot{}\cfoot{\thepage}\rfoot{}

%%% SECTION TITLE APPEARANCE
\usepackage{sectsty}
\allsectionsfont{\sffamily\mdseries\upshape} % (See the fntguide.pdf for font help)
% (This matches ConTeXt defaults)

%%% ToC (table of contents) APPEARANCE
\usepackage[nottoc,notlof,notlot]{tocbibind} % Put the bibliography in the ToC
\usepackage[titles,subfigure]{tocloft} % Alter the style of the Table of Contents
\renewcommand{\cftsecfont}{\rmfamily\mdseries\upshape}
\renewcommand{\cftsecpagefont}{\rmfamily\mdseries\upshape} % No bold!

%%% END Article customizations

%%% The "real" document content comes below...

\title{538 Riddler: 6 May 2022}
\author{Colin Parker}

\begin{document}
\maketitle

\section{Express}
Let $E_n$ be the expected number of stops from floor $n$ to floor 1, and we will calculate $E_{10}$.
Assuming a uniform distribution of floor probabilities, we are equally likely to stop at each floor on the way
down, which will add 1 to our total floor count. This gives the recursion

\begin{align*}
E_n & = 1 + \sum_{k=1}^{n-1} \frac{E_k}{n-1} \\
& = 1 + \frac{1}{n-1}\sum_{k=1}^{n-1} E_k; \; E_1 = 0
\end{align*}

Rearranging gives

\begin{align*}
(n-1)(E_n - 1) & = \sum_{k=1}^{n-1} E_k \\
& = E_{n-1} + \sum_{k=1}^{n-2} E_k \\
& = E_{n-1} + (n-2)(E_{n-1} - 1) \\
& = (n-1)(E_{n-1} - 1) + 1 \\
\iff E_n & = E_{n-1} + \frac{1}{n-1} \\
& = E_{n-2} + \frac{1}{n-2} + \frac{1}{n-1} \\
& = E_0 + \frac{1}{1} + \frac{1}{2} + \ldots + \frac{1}{n-1} \\
E_n & = \sum_{k=1}^{n-1}\frac{1}{k}
\end{align*}

The number we're looking for is $E_{10} = 1 + \frac{1}{2} + \ldots + \frac{1}{9} = \frac{7129}{2520} \approx 2.829$ floors. \\
\\

In general, the expected number of stopping floors, starting at floor $n$, is called the $(n-1)^{\text{th}}$
\emph{harmonic number} $H_{n-1}$. The asymptotic expansion allowing for easy approximation is

\begin{align*}
H_n & \sim \ln n + \gamma + \frac{1}{2n} - \sum_{k=1}^{\infty} \frac{B_{2k}}{2kn^{2k}} \\
E_n = H_{n-1} & \approx \ln (n-1) + \gamma + \frac{1}{2n - 2}
\end{align*}

In this expression, $\gamma \approx 0.5772156649$ is the Euler-Mascheroni constant and $B_k$ is the $k^{\text{th}}$ Bernoulli number, which can be used to add more terms to ensure more accurate approximation or faster convergence.

\section{Classic}

After generating lists of all three-digit (and four-, five-, and six-digit primes), I created a graph with connections between
the primes that formed adjacent rungs of a prime ladder (e.g. between 101 and 181). From there, repeated use of Dijkstra's
algorithm allowed me to find the shortest path (i.e. optimal prime ladder) between every pair of primes. Sorting these by
length, the longest chains are:

\begin{align*}
& \text{1-digit primes: 2 rungs} \\
& \text{2-digit primes: 4 rungs} \\
& \text{3-digit primes: 7 rungs} \\
& \text{4-digit primes: 9 rungs} \\
& \text{5-digit primes: 11 rungs} \\
\end{align*}

Unfortunately, there are 68906 6-digit primes, which means there are up to 2.3 billion connections to check and values
to store. This was too taxing on my RAM, so I couldn't find the answer in this case.

\end{document}
