% !TEX TS-program = pdflatex
% !TEX encoding = UTF-8 Unicode

% This is a simple template for a LaTeX document using the "article" class.
% See "book", "report", "letter" for other types of document.

\documentclass[11pt]{article} % use larger type; default would be 10pt

\usepackage[utf8]{inputenc} % set input encoding (not needed with XeLaTeX)

%%% Examples of Article customizations
% These packages are optional, depending whether you want the features they provide.
% See the LaTeX Companion or other references for full information.

%%% PAGE DIMENSIONS
\usepackage{geometry} % to change the page dimensions
\geometry{a4paper} % or letterpaper (US) or a5paper or....
% \geometry{margin=2in} % for example, change the margins to 2 inches all round
% \geometry{landscape} % set up the page for landscape
%   read geometry.pdf for detailed page layout information

\usepackage{graphicx} % support the \includegraphics command and options

% \usepackage[parfill]{parskip} % Activate to begin paragraphs with an empty line rather than an indent

%%% PACKAGES
\usepackage{amsfonts}
\usepackage{amsmath}
\usepackage{booktabs} % for much better looking tables
\usepackage{array} % for better arrays (eg matrices) in maths
\usepackage{paralist} % very flexible & customisable lists (eg. enumerate/itemize, etc.)
\usepackage{verbatim} % adds environment for commenting out blocks of text & for better verbatim
\usepackage{subfig} % make it possible to include more than one captioned figure/table in a single float
% These packages are all incorporated in the memoir class to one degree or another...

%%% HEADERS & FOOTERS
\usepackage{fancyhdr} % This should be set AFTER setting up the page geometry
\pagestyle{plain} % options: empty , plain , fancy
\renewcommand{\headrulewidth}{0pt} % customise the layout...
\lhead{}\chead{}\rhead{}
\lfoot{}\cfoot{\thepage}\rfoot{}

%%% SECTION TITLE APPEARANCE
\usepackage{sectsty}
\allsectionsfont{\sffamily\mdseries\upshape} % (See the fntguide.pdf for font help)
% (This matches ConTeXt defaults)

%%% ToC (table of contents) APPEARANCE
\usepackage[nottoc,notlof,notlot]{tocbibind} % Put the bibliography in the ToC
\usepackage[titles,subfigure]{tocloft} % Alter the style of the Table of Contents
\renewcommand{\cftsecfont}{\rmfamily\mdseries\upshape}
\renewcommand{\cftsecpagefont}{\rmfamily\mdseries\upshape} % No bold!

%%%Optionally increase matrix line spacing
\makeatletter
\renewcommand*\env@matrix[1][\arraystretch]{%
  \edef\arraystretch{#1}%
  \hskip -\arraycolsep
  \let\@ifnextchar\new@ifnextchar
  \array{*\c@MaxMatrixCols c}}
\makeatother

%%% END Article customizations

%%% The "real" document content comes below...

\title{538 Riddler: 13 May 2022}
\author{Colin Parker}

\begin{document}
\maketitle

\section{Express}
I first looked at letters used, sounds involved, state name length, and location without finding any kind of pattern.
At this point I thought of other unique identifiers for the states (postal code, flag, slogan, bird, flower, etc.), of which
postal code is by far the most commonly known and used. This, combined with Sherlock's clue: ``it's elementary'',
led me to try matching up state postal codes with symbols for chemical elements:

\begin{align*}
\text{Alabama = AL } & \text{= Al = Aluminum} \\
\text{Arkansas = AR } & \text{= Ar = Argon} \\
\text{California = CA } & \text{= Ca = Calcium} \\
\text{Colorado = CO } & \text{= Al = Aluminum} \\
\text{Florida = FL } & \text{= Fl = Flerovium (or perhaps phonetically Fluorine)} \\
\text{Georgia = GA } & \text{= Ga = Gallium} \\
\text{Indiana = IN } & \text{= In = Indium} \\
\text{Louisiana = LA } & \text{= La = Lanthanum} \\
\text{Maryland = MD } & \text{= Md = Mendelevium} \\
\text{Minnesota = MN } & \text{= Mn = Manganese} \\
\text{Missouri = MO } & \text{= Mo = Molybdenum} \\
\text{Montana = MT } & \text{= Mt = Meitnerium} \\
\text{Nebraska = NE } & \text{= Ne = Neon} \\
\text{North Dakota = ND } & \text{= Nd = Neodymium} \\
\text{Pennsylvania = PA } & \text{= Pa = Protactinium} \\
\text{South Carolina = SC } & \text{= Sc = Scandium}
\end{align*}

No other state's postal code is shared with an element's chemical symbol, so more than likely our criminal
mastermind is a chemist!

\section{Classic}

At any point while I'm playing the game, after rolling, I may have 0, 2, 3, or 4 die in the ``duplicate'' group; having a single
die is not possible. In addition, the probabilities of reaching a particular state on the next roll are completely determined
by my current state, not by any previous state or how long I've been playing. This is the Markov property, so we can, after
many probability calculations, determine the transition matrix and long-run behavior.

Let $a_n, b_n, c_n, d_n$ be the probabilities of getting 0, 1, 3, or 4 duplicates, respectively, after the $n^{\text{th}}$ roll.
For the first distribution we would typically use $a_0 = 1, \; b_0 = c_0 = d_0 = 0$, but in the subsequent game reaching
the state with 0 duplicates ends the game. Therefore we want to use the distribution \emph{after} the first roll:

$$\begin{bmatrix}[1.3] a_0 \\ b_0 \\ c_0 \\ d_0 \end{bmatrix} = \begin{bmatrix}[1.3] \frac{3}{32} \\ \frac{9}{16} \\ \frac{3}{16} \\ \frac{5}{32} \end{bmatrix}$$

By considering the possible outcomes for rerolling the duplicate die (which I'll not include for brevity's sake), we can find the
transition matrix of the Markov chain:

$$\begin{bmatrix}[1.3] a_n \\ b_n \\ c_n \\ d_n \end{bmatrix} = \begin{bmatrix}[1.3] 1 & \frac{1}{8} & \frac{3}{32} & 0 \\ 0 & \frac{3}{4} & \frac{9}{16} & 0 \\ 0 & \frac{1}{2} & \frac{3}{16} & 0 \\ 0 & \frac{1}{8} & \frac{5}{32} & 1 \end{bmatrix}^n \begin{bmatrix}[1.3] \frac{3}{32} \\ \frac{9}{16} \\ \frac{3}{16} \\ \frac{5}{32} \end{bmatrix}$$

In any Markov chain, the long-run behavior will be determined by stationary states (eigenvectors of the transition matrix
associated with eigenvalues of 1) and cyclic or periodic states (eigenvectors of the transition matrix associated with eigenvalues
$|\lambda| = 1$). All other eigenvalues have $|\lambda| < 1$, and therefore $\lim_{n \to \infty} \lambda^n = 0$ means that
they will have no effect in the long run.

This Markov process has two stationary states: having 0 or 4 duplicates means the game is over. Therefore, writing our
transition matrix eigenvalues as $\lambda_i$ and their associated eigenvectors as $\vec{v_i}$, we have immediately
that

$$\lambda_1 = \lambda_2 = 1; \; \vec{v_1} = \begin{bmatrix} 1 \\ 0 \\ 0 \\ 0 \end{bmatrix}; \; \vec{v_2} =
\begin{bmatrix} 0 \\ 0 \\ 0 \\ 1 \end{bmatrix}$$

Then, we can solve for the
eigenvalues by finding the other two roots of the characteristic polynomial $\operatorname{det} (M - \lambda I) = 0$. Finding their
associated eigenvectors amounts to solving the equation $M\vec{v_i} = \lambda_i \vec{v_i}$. Lastly, writing
$\begin{bmatrix} a_0 \\ b_0 \\ c_0 \\ d_0 \end{bmatrix} = \sum_{i=1}^4 c_i \vec{v_i}$, we can solve for the $c_i$
coefficients by solving a system of 4 equations in 4 variables. This results in

\begin{align*}
\begin{bmatrix} a_0 \\ b_0 \\ c_0 \\ d_0 \end{bmatrix} & = \frac{3}{7}\vec{v_1} + \frac{4}{7}\vec{v_2} - \frac{57}{136}\vec{v_3} + \frac{15}{3808}\vec{v_4} \\
\implies \begin{bmatrix} a_n \\ b_n \\ c_n \\ d_n \end{bmatrix} & = \frac{3}{7}\vec{v_1} + \frac{4}{7}\vec{v_2} - \frac{57}{136}\left(\frac{3}{4}\right)^n\vec{v_3} + \frac{15}{3808}\left(\frac{-5}{16}\right)^n\vec{v_4} \\
\implies \lim_{n \to \infty} \begin{bmatrix}[1.3] a_n \\ b_n \\ c_n \\ d_n \end{bmatrix} & = \begin{bmatrix}[1.3] \frac{3}{7} \\ 0 \\ 0 \\ \frac{4}{7} \end{bmatrix}
\end{align*}

Therefore the probability of winning (i.e. the probability of eventually having 0 duplicate die, 4 unique die) is $\frac{3}{7}$.

\end{document}
