% !TEX TS-program = pdflatex
% !TEX encoding = UTF-8 Unicode

% This is a simple template for a LaTeX document using the "article" class.
% See "book", "report", "letter" for other types of document.

\documentclass[11pt]{article} % use larger type; default would be 10pt

\usepackage[utf8]{inputenc} % set input encoding (not needed with XeLaTeX)

%%% Examples of Article customizations
% These packages are optional, depending whether you want the features they provide.
% See the LaTeX Companion or other references for full information.

%%% PAGE DIMENSIONS
\usepackage{geometry} % to change the page dimensions
\geometry{a4paper} % or letterpaper (US) or a5paper or....
% \geometry{margin=2in} % for example, change the margins to 2 inches all round
% \geometry{landscape} % set up the page for landscape
%   read geometry.pdf for detailed page layout information

\usepackage{graphicx} % support the \includegraphics command and options

% \usepackage[parfill]{parskip} % Activate to begin paragraphs with an empty line rather than an indent

%%% PACKAGES
\usepackage{amsfonts}
\usepackage{amsmath}
\usepackage{booktabs} % for much better looking tables
\usepackage{array} % for better arrays (eg matrices) in maths
\usepackage{paralist} % very flexible & customisable lists (eg. enumerate/itemize, etc.)
\usepackage{verbatim} % adds environment for commenting out blocks of text & for better verbatim
\usepackage{subfig} % make it possible to include more than one captioned figure/table in a single float
% These packages are all incorporated in the memoir class to one degree or another...

%%% HEADERS & FOOTERS
\usepackage{fancyhdr} % This should be set AFTER setting up the page geometry
\pagestyle{fancy} % options: empty , plain , fancy
\renewcommand{\headrulewidth}{0pt} % customise the layout...
\lhead{}\chead{}\rhead{}
\lfoot{}\cfoot{\thepage}\rfoot{}

%%% SECTION TITLE APPEARANCE
\usepackage{sectsty}
\allsectionsfont{\sffamily\mdseries\upshape} % (See the fntguide.pdf for font help)
% (This matches ConTeXt defaults)

%%% ToC (table of contents) APPEARANCE
\usepackage[nottoc,notlof,notlot]{tocbibind} % Put the bibliography in the ToC
\usepackage[titles,subfigure]{tocloft} % Alter the style of the Table of Contents
\renewcommand{\cftsecfont}{\rmfamily\mdseries\upshape}
\renewcommand{\cftsecpagefont}{\rmfamily\mdseries\upshape} % No bold!

%%% END Article customizations

%%% The "real" document content comes below...

\title{538 Riddler: 29 April 2022}
\author{Colin Parker}
%\date{} % Activate to display a given date or no date (if empty),
         % otherwise the current date is printed 

\begin{document}
\maketitle

\section{Express}

We can say a decimal expansion terminates if $\frac{1}{n} = \frac{p}{10^q}$ for some $p, q \in \mathbb{Z}$. This 
implies $np = 10^q = 2^q5^q$,
and therefore that $n$ is divisible only by 2 and 5. Conversely, if $n$ is divisible only by 2 and 5, we can write
$n = 2^a5^b, \; n \cdot 2^b5^a = 10^{a+b} \implies \frac{1}{n} = \frac{2^b5^a}{10^{a+b}}$ is a terminating decimal
expansion. Therefore the only values of $n$ for which the expansion of $1/n$ terminates are $n=2^a5^b$.
Taking inspiration from Euler's product formula for the Riemann zeta function (with $s=1$), we can write

\begin{align*}
\sum_{a, b \geq 0} \frac{1}{2^a5^b} & = \prod_{k \in \{2, 5\}}(1 + \frac{1}{k} + \frac{1}{k^2} + \ldots) \\
& = \prod_{k \in \{2, 5\}}\frac{1}{1-k^{-1}} \\
& = 2\cdot\frac{5}{4} = 2.5
\end{align*}

\section{Classic}

This problem was pretty impenetrable in the time I spent working on it. I could see that $x \approx 1/4$, just slightly
less, and that $P(\text{Nicks win}) \approx 1/8, P(\text{Naughts win}) \approx 3/8$, but I had a hard time proving
any of these things by induction, recursion, or raw calculation. Here is the code I used to obtain these numbers:

\begin{verbatim}
import numpy as np
from collections import defaultdict

def play_game(x):
  def play_pos(t0_score, t1_score, pos):
    score_prob = 0.5
    if pos: # i.e. team 1 (Naughts) has possession
      if t0_score > t1_score:
        score_prob += x
      elif t0_score < t1_score:
        score_prob -= x
      t1_score += (np.random.random() < score_prob)
    else: # team 0 (Nicks) has possession
      if t0_score > t1_score:
        score_prob -= x
      elif t0_score < t1_score:
        score_prob += x
      t0_score += (np.random.random() < score_prob)
    return (t0_score, t1_score, 1 - pos)

  game_state = play_pos(0, 0, 0)
  for _ in range(1, 50):
    game_state = play_pos(*game_state)

  return game_state[:-1]

results = dict()

for x in np.linspace(0.05, 0.45, 9):
  results[x] = defaultdict(int)
  for _ in range(10**4):
    result = play_game(x)
    if result[0] > result[1]: results[x][0] += 1
    elif result[0] < result[1]: results[x][1] += 1
    else: results[x]['Tie'] += 1

print(results)
\end{verbatim}



\end{document}
