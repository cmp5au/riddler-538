% !TEX TS-program = pdflatex
% !TEX encoding = UTF-8 Unicode

% This is a simple template for a LaTeX document using the "article" class.
% See "book", "report", "letter" for other types of document.

\documentclass[11pt]{article} % use larger type; default would be 10pt

\usepackage[utf8]{inputenc} % set input encoding (not needed with XeLaTeX)

%%% Examples of Article customizations
% These packages are optional, depending whether you want the features they provide.
% See the LaTeX Companion or other references for full information.

%%% PAGE DIMENSIONS
\usepackage{geometry} % to change the page dimensions
\geometry{a4paper} % or letterpaper (US) or a5paper or....
% \geometry{margin=2in} % for example, change the margins to 2 inches all round
% \geometry{landscape} % set up the page for landscape
%   read geometry.pdf for detailed page layout information

\usepackage{graphicx} % support the \includegraphics command and options

% \usepackage[parfill]{parskip} % Activate to begin paragraphs with an empty line rather than an indent

%%% PACKAGES
\usepackage{amsfonts}
\usepackage{amsmath}
\usepackage[backref,pdfpagemode=FullScreen,colorlinks=true]{hyperref}
\usepackage{booktabs} % for much better looking tables
\usepackage{array} % for better arrays (eg matrices) in maths
\usepackage{paralist} % very flexible & customisable lists (eg. enumerate/itemize, etc.)
\usepackage{verbatim} % adds environment for commenting out blocks of text & for better verbatim
\usepackage{subfig} % make it possible to include more than one captioned figure/table in a single float
% These packages are all incorporated in the memoir class to one degree or another...

%%% HEADERS & FOOTERS
\usepackage{fancyhdr} % This should be set AFTER setting up the page geometry
\pagestyle{plain} % options: empty , plain , fancy
\renewcommand{\headrulewidth}{0pt} % customise the layout...
\lhead{}\chead{}\rhead{}
\lfoot{}\cfoot{\thepage}\rfoot{}

%%% SECTION TITLE APPEARANCE
\usepackage{sectsty}
\allsectionsfont{\sffamily\mdseries\upshape} % (See the fntguide.pdf for font help)
% (This matches ConTeXt defaults)

%%% ToC (table of contents) APPEARANCE
\usepackage[nottoc,notlof,notlot]{tocbibind} % Put the bibliography in the ToC
\usepackage[titles,subfigure]{tocloft} % Alter the style of the Table of Contents
\renewcommand{\cftsecfont}{\rmfamily\mdseries\upshape}
\renewcommand{\cftsecpagefont}{\rmfamily\mdseries\upshape} % No bold!

%%%Optionally increase matrix line spacing
\makeatletter
\renewcommand*\env@matrix[1][\arraystretch]{%
  \edef\arraystretch{#1}%
  \hskip -\arraycolsep
  \let\@ifnextchar\new@ifnextchar
  \array{*\c@MaxMatrixCols c}}
\makeatother

%%% END Article customizations

%%% The "real" document content comes below...

\title{538 Riddler: 20 May 2022}
\author{Colin Parker}

\begin{document}
\maketitle

\section{Express}

When Amare is at a radius $r$ from the center of the web, crawling radially away puts him at a distance $r + 1$ and
crawling tangentially creates a right triangle with legs 1 and $r$, putting him a distance $\sqrt{r^2 + 1}$ from the
center. As this is nonlinear, we can't simply put in the expected distance for $n$ steps in order to get the
expected distance for $n + 1$ steps; we have to enumerate the possibilities. Luckily for us, Amare must be 1 inch
away from the center after the first step, then there are 3 more steps to take for a total of only $2^3 = 8$ possibilities.
These are, in increasing order, $\{ 2, \sqrt{4 + 2\sqrt2}, \sqrt{10}, 1 + \sqrt3, 1 + \sqrt5, 2 + \sqrt2, 4 \}$,
and averaging them gives an expected distance of 2.9509 inches from the center of the web. 

\section{Classic}

We can imagine that our point of view relative to Sagittarius A* corresponds to a point on a sphere with the black
hole at its center. The accretion disk would be at its ``equator'' and the line perpendicular to it through the black hole
hitting the sphere at its ``poles''. Then, because our point is a random variable under a uniform distribution on the
surface of the sphere, the probability that it is within $10^{\circ}$ of one of the poles is the fraction of these areas
divided by the total sphere area. This is given by

$$\frac{2A}{4\pi r^2} = \frac{\Omega}{2\pi} = 2\sin^2\left(\frac{10^{\circ}}{2}\right) = 2\sin^2(5^{\circ}) \approx 0.01519$$

Where $A$ is the area of one of these two ``polar caps'' and $\Omega$ is the solid angle generated by the $10^{\circ}$
central angle (\href{https://en.wikipedia.org/wiki/Steradian}{Steradian on Wikipedia} for reference).

\end{document}
